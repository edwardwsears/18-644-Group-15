\documentclass[pageno]{jpaper}


\usepackage[normalem]{ulem}

\begin{document}

\title{
18-644 Fall 2013 Project Proposal: GPS Bus Tracker}
\author{Edward Sears <esears@andrew.cmu.edu> \and Grant Skudlarek <gskudlar@andrew.cmu.edu> \and Kory Stiger <kstiger@andrew.cmu.edu>}

\date{}
\maketitle

\thispagestyle{empty}

\begin{abstract}
In this paper, we will propose a novel way to track busses and 
accurately predict arrival times.  We start by formally stating
the problem and its importance.  We then list the competition 
and their subsequent limitations.  Lastly, we describe our 
idea and list the various benefits associated with implementing 
our idea.
\end{abstract}

\section{The Problem}

Public transporation can be an important asset to any city. If widely 
used, it can improve air quality, reduce traffic congestion, and more 
importantly provide transportation for low income families \cite{Criden08} and those who 
do not have cars.  However, in Pittsburgh (and many other cities across the country) 
the public transportation is very unreliable.  The bus system posts 
schedules, but can rarely stick to them.  This can cause up to an 
hour delay between the scheduled time and the bus arrival time.  For 
many people, this uncertainty makes the bus system practically unusable.  
Because of this, fewer and fewer people are using public transportation 
regularly.  This results in the bus company earning less revenue and 
consequently reducing the number of bus routs.  If this trend continues, 
bus companies will not be able to stay in business, leaving many people 
without any mode of transportation.

\section{Importance}

Making public transportation more user friendly could entice many more 
people to use it.  This would not only increase revenues for the bus 
company, but would create a valuable resource for the residents of 
the city.  More people could rely on the bus system on a regular basis 
and could increase the quality of living in the city in general.

\section{Competition}

\begin{table}[h!]
  \centering
  \begin{tabular}{|l|l|}
    \hline
    \textbf{Name} & \textbf{Limitation}\\
    \hline
    \hline
    Google Maps & Only posts scheduled times \\
    \hline
    CTA Bus Tracker & Chicago specific \\
        &(not easily expandable)\cite{CTA} \\
    \hline
    Tiramisu & Relies on individuals \\
                    &to report bus activity\\
                    &(crowdsourced) \cite{Tiramisu}\\
    \hline
    Maimi-Dade  & Miami Specific,\\
    Bus Tracker & Only tracks a single route \\
    \hline
  \end{tabular}
  \label{table:formatting}
\end{table}

\section{Our Idea}

We wish to predict bus arrival times via GPS tracking of all buses in Pittsburgh.
GPS modules placed on each bus would transmit the buses coordinates to a server
application via cell networks. The server could then serve bus location data
to users via the internet, who may be accessing the data through their mobile
devices.

While GPS data will let users know where buses are, it may not let them
know how long it will take for the bus to arrive at their stop. The user will
not be able to know if the bus is stuck in traffic, or if it will arrive within
a reasonable amount of time. The Pittsburgh bus schedule already shows how long
it usually takes for a bus to travel between stops. In the absence of traffic,
this data can be used to determine when the bus will arrive, based on its current
position along the route.

In order to give users accurate results regardless of conditions, traffic patterns
must be gathered in real time, or be based on a record of past traffic patterns.
One approach which we are considering taking is the use of Google traffic data.
By querying Google traffic data in real time along the buses route, the
appropriate traffic delay can be added based on the situation. The other approach
we are considering would be to accumulate a model of traffic patterns over a 
time period. This data, accumulated from past bus runs, would allow arrival time
on a route to be predicted during a certain time of day. The actual position
data and journey times of buses could be used to make the model more accurate. 
For the time period used for the model, longer time periods would allow more 
accurate predictions, but shorter time periods allow for faster adaption and 
deployment. We've determined that the best compromise would be week-long cycles.

The hardware module would require a GPS receiver, a processing unit, and a
cellular modem. For our proof of concept, we will use an Arduino for the
processing unit. For the modem component, we will use an inexpensive cell
phone connected to a circuit board which can access the cellular modem
via the phone serial port.

\section{Benefits of Our Idea}

Our idea is more effective than current methods for 3 main reasons:
\begin{itemize}
\item Reliability: Because our system uses real hardware instead of 
        relying on crowdsourcing techniques, the quality of service
        does not vary by time of day or by popularity.
\item Easy Set Up: Our system can leverage the bus route data that 
        is curretly present in Google Maps.  Because of this, 
        our system can be easily ported to different cities 
        and different public transportation modes.
\item Accurate Prediction:  Our system will have the ability to 
        accurately predict arrival times for busses by learning 
        trends.  These trends will be based upon time of day as 
        well as other historical factors.  This can create a much 
        more intellegent prediction mechanism compared to other 
        bus tracking services.
\end{itemize}

\bstctlcite{bstctl:etal, bstctl:nodash, bstctl:simpurl}
\bibliographystyle{IEEEtranS}
\bibliography{references}

\end{document}

