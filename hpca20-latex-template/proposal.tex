\documentclass[pageno]{jpaper}


\usepackage[normalem]{ulem}

\begin{document}

\title{
18-644 Fall 2013 Project Proposal: GPS Bus Tracker}
\author{Edward Sears \and Grant Skudlarek \and Kory Stiger}

\date{}
\maketitle

\thispagestyle{empty}

\begin{abstract}
\end{abstract}

\section{The Problem}

    Public transporation can be an important asset to any city. If widely 
used, it can improve air quality, reduce traffic congestion, and more 
importantly provide transportation for those who do not have cars.  
However, in Pittsburgh (and many other cities across the country) 
the public transportation is very unreliable.  The bus system posts 
schedules, but can rarely stick to them.  This can cause up to an 
hour delay between the scheduled time and the bus arrival time.  For 
many people, this uncertainty makes the bus system practically unusable.  
Because of this, fewer and fewer people are using public transportation 
regularly.  This results in the bus company earning less revenue and 
consequently reducing the number of bus routs.  If this trend continues, 
bus companies will not be able to stay in business, leaving many people 
without any mode of transportation.

\section{Importance}

-increase in revenue for bus companies

\section{Competition}

\begin{table}[h!]
  \centering
  \begin{tabular}{|l|l|}
    \hline
    \textbf{Name} & \textbf{Limitation}\\
    \hline
    \hline
    Google Maps & Only posts scheduled times \\
    \hline
    CTA Bus Tracker & Chicago specific \\
        &(not easily expandable)\\
    \hline
    Tiramisu & Relies on individuals \\
                    &to report bus activity\\
                    &(crowdsourced) \\
    \hline
    Maimi-Dade  & Miami Specific,\\
    Bus Tracker & Only tracks a single route \\
    \hline
  \end{tabular}
  \label{table:formatting}
\end{table}

\section{Our Idea}


\section{Benefits of Our Idea}

-Reliable, uses hw not crowdsourcing
-Easily set up in any city (if routs tied into google maps)
-Cost effective


\bstctlcite{bstctl:etal, bstctl:nodash, bstctl:simpurl}
\bibliographystyle{IEEEtranS}
\bibliography{references}

\end{document}

